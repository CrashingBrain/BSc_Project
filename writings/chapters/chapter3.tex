\section{The abstraction through random variables}
    We have used many times the idea of \textit{factoring out} the enemy in the previous chapter, but we never actually expressed what we meant with that.
    In order to make reasoning like this it is needed to model the concepts of \textit{message} and \textit{information} in a computation-able way.
    C. Shannon studied this problem and published his results in \cite{Shannon49} which is now the foundation of modern information theory and cryptography.
    
    Let us suppose that over a certain alphabet exist messages $M_1,M_2,\ldots M_n$ and each message has a probability $P(M_i)$ to be chosen (i.e. transmitted).
    Each message $M_i$ is encrypted into its counterpart $E_i$.
    An enemy intercepts $E_i$ and can therefore calculate the probability of message $M_i$ corresponding to the received encrypted version; namely the conditional probability $P(M_i|E_i)$.
    Shannon states that to obtain perfect secrecy of the message $P(M|E)$ must equal $P(M)$ for all $E$ and all $M$.
    Using Bayes' theorem we can formulate
    \begin{equation}
    	P(M|E) = \frac{P(M)P(E|M)}{P(E)}
    \end{equation}
    which gives the sufficient and necessary condition that $P(E|M)=P(E)$ for perfect secrecy. 
    The probability of obtaining $E$ a priori must be the same as receiving $E$ if message $M$ was encrypted.
    This translates into the case where intercepting the encrypted message gives the enemy no information. 
    
    Now imagine the message is transmitted by a satellite to Alice and Bob.
    Eve is also listening. This is an example of a public channel.
    We end up with three versions of the message\footnotemark : Alice's version $X$, Bob's $Y$ and Eve's version $Z$.
    To express the whole space of combinations of possible messages, we need the joint probability $P_{XYZ}$.
    Then the idea of "factoring out Eve" takes the meaning of obtaining 
    \begin{equation}
    	P_{XYZ} = P_{XY}\cdot P_Z
    \end{equation}
    so that the marginal of $P_{XYZ}$ over $Z$ --- i.e. the part of the distribution owned by Eve --- is now product with variables $X$ and $Y$. $Z$ must be independent from $X$ and $Y$.
    
    \footnotetext{Ideally the three messages are identical, but just consider that the message is sent through a noisy channel: each receiver will have slight variations of the message, hence the distinction.}
    
    From probability theory comes information theory, which provides us with useful measures and rules to operate on those probabilities.
    The most important to us are the entropy $\Ent (X)$, the correlation and mutual information $\I(X;Y)$ of random variables and the marginal $P_X$ of joint probabilities $P_{XY}$.
    
    
    
%    In classical information theory a message is defined as a repetition of an experiment on random variables with a (joint) probability distribution $P_{X_1X_2\ldots X_n}$.
%    The range of the random variable $X$ is the size of the alphabet with which we are communicating.
%    For example if the message is written in English the alphabet is composed of $26$ letters, or $62$ if we include numbers and distinguish upper and lower case letters.
%    The probability of a word in a message can be given, for example, by the frequency of that word in the English language.
%    Moreover, the probability of a word (or letter) to follow another can be dependent on the previous realizations of $P$.
%    Probability theory provides rules and operations to manipulate and observe these probability distributions.
%    The most important for us are the entropy $\Ent (X)$, the correlation and mutual information $\I(X;Y)$ of random variables and the marginal $P_X$ of joint probabilities $P_{XY}$.
    
\section{Local operations and public communication (LOPC)}
    By local operations and public communication we mean operations carried out on $P_{XYZ}$ representing our variables.
    We can mix different distributions together or trace out the marginal.
    Communicating over a channel is, in nature, noisy. 
    That noise is also a form of operations on the probability distribution.
    Operations can also be carried out directly by the parties.
    This is of more interest because we have control over those operations.
    
    Take for example the Diffie-Hellman method illustrated in section \ref{Diffie}.
    Steps $1$ and $4$ are \emph{public communication} and steps $2$,$3$ and $5$ are \emph{local operations}.
    Local operations are conducted privately, meaning that everything that happens is only accessible to the party conducting the operation.
    Other parties can not know what a local operation involved.
    Public communication is everything that is communicated in clear by parties, or that an eavesdropper can intercept.
    The totality of what an enemy knows from a protocol --- apart from the functioning of the protocol itself --- comes from public communication.
    Through public communication Alice and Bob can also send instructions on what to do in their local operations, like in BB84.
    
    
\section{Secret-key rate} \label{seckeyrate}
    The secret-key rate $\keyrate{X}{Y}{Z}$ is roughly a quantification of the maximal amount of correlated bits between Alice and Bob extractable from an arbitrarily large number of independent realizations of a distribution $P_{XYZ}$.
    It was introduced by Maurer in \cite{Maur93} to prove lower bounds on the achievable size of a key shared by Alice and Bob in secrecy.
    It can be seen as the classical analog of the \textit{distillable entanglement} in \cite{BDS96}.
    \begin{definition}\cite{Maur93, RW03}
    Let $P_{XYZ}$ be a joint probability distribution. The secret-key rate $\keyrate{X}{Y}{Z}$ of $X$ and $Y$ with respect to $Z$ is the largest $R\in \mathbb{R}$ such that for all $\epsilon > 0$ there exists a protocol, that involves a sufficiently large number $N$ of realizations of $X^N$ of $X$ and $Y^N$ of $Y$, that satisfies:
    Alice and Bob compute, at the end of the protocol, random variables $S_A$ and $S_B$, respectively, with range $\mathcal{S}$ such that there exists a random variable $S$ with the same range and
    \begin{equation*}
    	\Ent(S) = \log |S| \geq RN\; ,
    \end{equation*}
    \begin{equation*}
    	P[S_A=S_B=S]>1-\epsilon\; ,
    \end{equation*}
    \begin{equation*}
    	\I(S;CZ^N) < \epsilon
    \end{equation*}
	Here $C$ is the totality of the protocol public communication; $\I(S;CZ^N)$ is the mutual information between the secret key and what the eavesdropper Eve knows (ref. \ref{mutInfo}).
    \end{definition}
    
    The secret-key rate is a useful measure of the amount of extractable secrecy --- hence, key bits --- from a protocol with LOPC.
    It would be ideal to be able to express it as a function $\keyrate{X}{Y}{Z} = S(P_{XYZ})$.
    Instead, it can be bounded by two functions\cite{Maur93}
    \begin{equation}\label{eq:skrbound1}
    	\keyrate{X}{Y}{Z} \leq \min[\I(X;Y),\, \I(X;Y|Z)]
    \end{equation}
    \begin{equation}\label{eq:skrbound2}
    	\keyrate{X}{Y}{Z} \geq \max[\I(Y;X) - \I(Z;X),\, \I(X;Y) - \I(Z,Y)]
    \end{equation}
    Those are not strict bounds. In fact in Eq. \ref{eq:skrbound2} the secret-key rate can be positive even when the right-hand side is negative.
    
    
    
