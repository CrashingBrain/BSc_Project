\label{ch:two}
\section{What is a shared key?}
	The first step towards understanding \emph{bound information} is looking at the end product of a key exchange. 
	The secret key is what we want to obtain from a protocol, so we must understand what we are after.
	What is then a shared key? 
	How do we define a common secret shared between Alice and Bob than can be used formally later on? 
%    	\subsection{Common Secret} \label{commonsecret}
    	
Intuitively a common secret is a piece of information (i.e. \textit{bits} of information) known to trusted parties --- for example Alice and Bob --- and to none else. 
In an environment where we allow the presence of an eavesdropper Eve that makes observation on the communication reaching such state is not always trivial. 
There exist methods and protocols to generate such secrets, even from nothing, although they reach different levels of secrecy. A notable one is the famous Diffie-Hellman method to generate a common cryptographic key \cite{DH76} . 
We discuss in section \ref{comparison}, while still secure, it is not \textit{information theoretical secure}.
Here we provide a mathematical definition of a common secret that makes use of concepts that will be explained later in chapter \ref{ch:four}.
    	\begin{definition}
	Let $X,Y,Z,S$ be random variables on the same range $\mathcal{X}$. Let $X$ be owned by Alice, $Y$ by Bob and $Z$ by Eve. Then if
  \begin{equation} \label{eq:common}
	  P[X=Y=S] > 1 - \epsilon \tag{common}
	\end{equation}
	\begin{equation} \label{eq:secret}
	  \I(X;Z) = 0 \: \wedge \: \I(Y;Z) = 0 \tag{secret}
  \end{equation}
for all $\epsilon > 0 $ we say Alice and Bob share a common secret.
    	\end{definition}
The first part defines the \textit{common} property: $X$ and $Y$ --- Alice and Bob's variables in the system --- must be asymptotically the same, i.e. the probability that they are the same comes arbitrarily close to $1$ for an arbitrarily large number of realization.
The second part states that the amount of information Eve can gather about $X$ and $Y$, through it's realization of $Z$, is zero.
\section{The analogy with entanglement}
	A fascinating feature that arises from quantum mechanics is quantum entanglement. As Einstein, Podolsky and Rosen pointed out almost a century ago \cite{einstein1935}, 
	the measurement of entangled states defies the classical understanding of a state. 
	They concluded that the theory is incomplete and has to be replaced.
	Bell responded to that with \emph{non-locality}\cite{Bell64}: there are probability distributions for which there is no local hidden variable model, i.e. exactly what EPR are chasing for.
	As quantum mechanics gives rise to such non-local probability distributions we cannot hope to find a "local hidden variable model" replacing quantum mechanics.
	
	For a quantum system to exhibit non-locality, entanglement is necessary.
%	Entangled particles remain connected so that actions on one affect the other, even when separated by great distances.
	Consider two pure quantum states $\ket{\psi}_A$ and $\ket{\phi}_B$.
	The composite system of the two states is
	\begin{equation}\label{eq:separable}
		\ket{\psi}_A \otimes \ket{\phi}_B = \ket{\psi\phi}_{AB}
	\end{equation}
	States that can be represented as in Eq. \ref{eq:separable} are called \emph{separable}.
	Not all states are separable. Non-separable states are called \emph{entangled}.
	Consider now the state
	\begin{equation}
		\frac{1}{\sqrt{2}}\left(\ket{0}_A\otimes\ket{1}_B - \ket{1}_A\otimes\ket{0}_B\right) = \ket{\Psi^{-}}_{AB}
	\end{equation}
	for it there is no decomposition into (indices are omitted)
	\begin{multline}
		\frac{1}{\sqrt{2}}\left( (\alpha_0\ket0+\alpha_1\ket1)\otimes (\beta_0\ket0 + \beta_1\ket1) \right) \\
		= \frac{1}{\sqrt{2}} \left( \alpha_0\beta_0\ket{00} + \alpha_0\beta_1\ket{01} + \alpha_1\beta_0\ket{10} + \alpha_1\beta_1\ket{11} \right)
	\end{multline}
	that satisfies 
	\begin{align}
	\alpha_0\beta_0 = \alpha_1\beta_1 & = 0 \\
	\alpha_0\beta_1 & = 1\\
	\alpha_1\beta_0 & = -1
	\end{align}
	This is an entangled state\footnote{This particular state is called "singlet" and is one of the four Bell's basis presented in \cite{Bell64}}.
	
	Furthermore, if Alice measures to have $\ket{0}$ on her part of the system, then Bob, using the same measurement basis, by non-locality, will measure $\ket{1}$.
	The other option of measuring $\ket1$ for Alice and $\ket0$ for Bob is also equally probable.
	Alice and Bob's values are always (anti-)correlated, regardless of which of the measurements is obtained (which is random).

%	Let a joint quantum system between Alice and Bob be entangled.
%	Tracing out the part of the system only for Alice (respectively for Bob) will give a completely mixed state. 
%	Since entanglement is a phenomenon that appears non-locally, it can be measured only on the whole system, i.e. one has to have access to both the states.
%	This makes the two states \emph{inseparable}. 
%	A complete (anti-)correlation exists then between maximally entangled states when measured in the same basis.
%	 This allows Alice and Bob to encode a message onto the quantum state and use it as a variable to share a secret\footnotemark . 
%	If Alice measures to have $\ket{0}$ on her part of the system, then Bob, using the same measurement basis, by non-locality, will measure $\ket{1}$\\
	
	Quantum entanglement posseses one more feature that classical correlation does not have: the monogamy of entanglement \cite{KW04}. 
	As Koashi and Winter state in their paper a fundamental difference is that classical correlation can be shared, while quantum entanglement can not. 
	This translates to the case where an eavesdropper Eve listens to the message exchange between Alice and Bob: in the classical communication there is no direct way for Alice nor Bob to know that Eve is listening (i.e. \textit{shares the correlation}), while in the second case Eve breaks the existing correlation between Alice and Bob.
These two aspects of quantum entanglement --- correlation and monogamy --- give a valid framework for the establishment of a private channel between parties.
	
	
	\footnotetext{A quantum state in quantum mechanics describes a single and isolated quantum system. This can be for example an electron or a photon. For our purposes, a quantum state is always abstracted as a \emph{qubit} or multiple qubits, as described in appendix \ref{App:appendixB}}
	\footnotetext{There are known protocols that achieve that, for example in \cite{Ekert91} or \cite{BB84}}
%   	 \begin{figure}[h]
%			\centering
%			%% Little schematics showing the origin of entanglement
%% from the linear theory of QM and tensor product

\begin{tikzpicture}[scale=0.6]
  \node[colorbox=red]                      (lin)  {Linearity};
  \node[colorbox=magenta, below=.5cm of lin] (tp)  {Tensor product};
  \node[colorbox=blue, below=.5cm of tp]   (en)  {Entanglement};
  \draw[conn=2pt, densely dotted] (tp) to (lin);
  \draw[conn=2pt, densely dotted] (en) to (tp);
\end{tikzpicture}
%			\caption{origin of entanglement via linearity}
%		\end{figure}
		
\section{Examples of key exchange}
	Exchanging keys for encryption was once done \textit{physically}, requiring the parties to meet and assure that no eavesdropper was present.
	Modern cryptographic systems make use of protocols over telecommunication channels. 
	In both cases the result at the end is that the trusted parties leave (or terminate the protocol) with a bit of information that they know it will be known only to them.
	Here we present examples for both classical and quantum mechanical channels and compare them.
	The intention is to compare them and discuss on the different level of security -- computational, physical and information theoretical --- one might achieve with the correct implementation of one of those.
		\subsection{The Diffie-Hellman key exchange}
		\label{Diffie}
		% Explain only how the protocol works, what is based on, what are its bounds, how it can be attacked (ideally).
		% Don't dive into Maurer violations, that will be covered in chapter [3] (XX)
	
		A famous and widely used method for the exchange of cryptographic keys is the Diffie-Hellman (DH) key-exchange method.
	The whole process can be summarized in five basic steps \cite{DH76}:
	\begin{enumerate}
		\item Alice and Bob \emph{publicly} communicate and agree on two numbers, that will serve as basis for the computations.
		\item Each party generates \emph{locally} a personal and distinct secret ($s_A$ and $s_B$) without ever communicating it .
		\item They mix their own secret with the common agreed basis, producing a result $R_A$ and $R_B$. The mathematical properties of this operation make it so it is computational infeasible to go back and retrieve the secrets $s$ from $R$.
		\item Both parties exchange \emph{publicly} their result. Each party now know both the result of the other and their own.
		\item Each party applies their secret to the received $R$. The outputs are equal for Alice and Bob so they can use this result as a common secret to create a key.
	\end{enumerate}	 
	
	The parts exchanged over the public channel --- the ones that Eve knows --- are only the mutually agreed base and the two partial mixtures. 
	It can be proven that those two elements alone give no information about the complete shared secret and that it is virtually impossible (within reasonable amount of time and use of resources) to obtain the correct final product with only those two.\\  
	
	\subsection{The BB84 protocol}
	Protocols for the exchange of keys over a quantum channel have been invented \cite{BB84, Ekert91}.
	These protocols work on the underlying physics of quantum mechanics.
	Alice and Bob need then to have access to a quantum channel to exchange quantum states
	\footnote{A quantum channel is anything that can carry quantum states between two points. For example an optical fiber that carries photons.}.
	Here follows the BB84 protocol as described in \cite{NC10} :
		\begin{enumerate}
			\item Alice chooses $(4+\delta )n$ data bits.
			\item Alice chooses a random $(4+\delta )n$-bit string $b$. She encodes each data bit as $\{\ket{0},\ket{1}\}$ if the corresponding bit of $b$ is $0$ or with the diagonal basis $\{\frac{\ket{0} + \ket{1}}{\sqrt{2}},\frac{\ket{0} - \ket{1}}{\sqrt{2}}\}$ if $b$ is $1$.
			\item Alice sends the resulting state to Bob.
			\item Bob receives the $(4+\delta )n$, announces \emph{publicly} this fact, and measures each qubit in the $X$ or $Z$ basis at random.
			\item Alice announces \emph{publicly} $b$.
			\item Alice and Bob discard any bits where Bob measured a different basis than Alice prepared. With high probability, there are at least $2n$ bits left (if not, abort the protocol and restart). They keep $2n$ bits.
			\item Alice selects a subset of $n$ bits that will to serve as a check on Eve's interference, and tells Bob which basis she selected.
			\item Alice and Bob announce \emph{publicly} and compare the values of the $n$ check bits. If more than an acceptable number disagree, they abort the protocol.
			\item Alice and Bob perform information reconciliation and privacy amplification on the remaining $n$ bits to obtain $m$ shared key bits.
		\end{enumerate}

	\subsection{The one-time Pad}
		%this is not really a method to share a common secret, since you have 
		% to already start with a secret.. <- point that out!
		
		% example of utilization of the key after generation
		% info theoretical key exchanged with OTP over already existing channel
		The one-time pad is not a key exchange method, but a technique to encrypt a message once the key is obtained.
		Here is presented as an example of utilization of a key previously exchanged with some other secure algorithm. 
		The message obtained from the OTP is as secure as the method that produced the key.
		This means that if the key is provided to be information theoretical secure, then Eve is \emph{completely factored out} from the information contained in the message.
		Shannon proved in \cite{Shannon49} the perfect security of OTP.
		Figure \ref{Fig:OTP} illustrates an example over a 5-bits string.
		It is important to notice that in order for to OTP to function correctly the key has to be perfectly random and at least of the same size of the message.
		
		\begin{figure}[h!]
			\centering
			% OTP 
\begin{tikzpicture}
\draw[thick,dashed,rounded corners=.7cm] (-2.5,-3.) rectangle (2.5, 3);
\node[colorbox=white] (plc) at (0,0) {Placeholder};
\end{tikzpicture}



			\caption{The one-time pad. In order to get true security the key should change each time for each message.}
			\label{Fig:OTP}
		\end{figure}
\section{A comparison between securities}\label{comparison}
    A point can be made comparing these different way of establishing privacy.
    The majority of cryptographic systems used are built on computational complexity security. 
	When Shannon laid out the basis of information theory in \cite{Shannon49} in 1949, he affirmed that the highest level of security (information theoretical security) can only be achieved by sharing a secret key from the beginning.
	Maurer later expanded this concept saying that it is not possible to obtain an information theoretical secure key from just a protocol through local operations and public communications\cite{Maur93}. A starting initial correlation must be provided.\\
	 
	 Even though the Diffie-Hellman method does start from nothing, it does not violates Maurer's statement.
	 The key produced by DH is not information theoretical secure. 
	 Step $3$ enumerated above relies on computational complexity.
	 The difficulty of breaking this step --- thus accessing the correlation between Alice and Bob --- is bounded only by the length of the number chosen one one side and the computational power available to the adversary on the other.
	 This means, for example, that one cannot use a key obtained through Diffie-Hellman (DH) to generate an information theoretical secure key: 
	 as DH starts from nothing, a protocol $R$ constructed to start as DH and producing information theoretical secure key with LOPC will violate the statement that no key can be crated by LOPC alone.
	 The BB84 protocol works differently, but also does not violate Maurer.
	 Albeit Alice and Bob share, at the end of the protocol, a key starting from nothing, and the secret is in theory information theoretical secure --- it can be arbitrarily close ---, it does not so with public (classical) communication.
	 Alice utilizes a quantum channel to pass the states to Bob, which is not public because of the monogamy of entanglement.
	 The communication part in BB84 over a classical channel serves only to check for the presence of an enemy Eve, not to modify the secret --- which is already transferred at this point.  
	 Thus the key is created only through a quantum channel, which is not covered by Maurer's claim.

%    The so-called cryptographic functions are functions that are easy to compute in one way, but have a much higher complexity the other way round.\\
%    This is the case for example for the Diffie-Hellman method. 
%	The security in this method relies mainly on step 3 (listed above).    
%    Here an action as $ R_A = g^{s_A} \bmod p $ is performed, where $g$ and $p$ are the public common basis agreed beforehand. 
%	To get back to $s_A$  one will need to find the prime factors of $R_A$, which is a known hard problem. 
%	It is not impossible nevertheless. 
%	The difficulty of breaking this step --- thus accessing the correlation between Alice and Bob --- is bounded only by the length of the number chosen one one side and the computational power available to the adversary on the other.\\
%	
%	The BB84 protocol works differently. 
%	It takes advantage of the monogamy of entanglement. 
%	If Eve was able to wire-tap the quantum channel, she would disrupt the correlation between Alice and Bob.
%	In this sense the correlation between Alice and Bob is much more protected, since the eavesdropper Eve is factored out from the entangled state $\rho = \rho_{AB}\otimes\rho_E$.
%	However, the entanglement in the protocol is only used to spot the presence of Eve over a large number of repetitions of the experiment.
%	While the amount of information that Eve gets can be bounded by some $\epsilon$, she is not yet \emph{completely} factored out.\\
	 
	
    
\section{The equivalent of CKA in QM}
	What we were trying to do was to factor out Eve from Alice and Bob's point of view.
	The goal of classical key agreement (CKA) is also to create a private correlation between Alice and Bob, from one that also includes Eve.
	We already stated that entangled states can provide this level of privacy \cite{Ekert91}.
	However pure entangled states are very fragile and do not occur in nature.
	The more general quantum state is a mixed state.
	A mixed state is a convex mixture of pure states with their probability.
	A mixed state is represented by a density matrix, defined as a positive, trace-$1$ operator.
	By the spectral theorem they can then be wrote in the form of
	\begin{equation}
		\rho = \sum_i p_i \ketbra{\psi_i}{\psi_i} ,\quad \ket{\psi_i}: \text{ state with probability } p_i
	\end{equation}
	and interpreted as a statistical 
	Assume that Alice and Bob start with a state $\ket{\psi_{ABE}}$, which is also mixed with a part owned by Eve. 
	Is there a way to factor her out the state, so that they obtain
	\begin{equation}
		\rho_{AB} = \Tr_E \kb{\psi_{ABE}}{\psi_{ABE}}
	\end{equation}
	with
	\begin{equation}
		\rho_{AB}\otimes\rho_E
	\end{equation}
	after a series of local operations and classical communication (LOCC)?
	Quantum distillation is a process that allows that. \label{distillation}
	If Alice and Bob have maximally entangled states after distillation then they know --- by the monogamy of entanglement --- that Eve is factored out.
	In other words, the state Alice and Bob have is product with the environment and entangled among them.
	The joint probability distribution $P_{ABE}$ falls similarly into a product $P_{AB}\cdot P_E$ after a CKA protocol.
	However from within $P_{AB}$ we cannot decide whether Eve is factored out or not in classical key agreement.
    
    \begin{figure}[h]
    	\centering
    	% Image showing intuition process
% Sort of justify how and why 
% bound entanglement and bound information
% should share analogies

\begin{tikzpicture}
\tikzstyle{vecArrow} = [thick, decoration={markings,mark=at position
   1 with {\arrow[semithick]{open triangle 60}}},
   double distance=1.4pt, shorten >= 5.5pt,
   preaction = {decorate},
   postaction = {draw,line width=1.4pt, white,shorten >= 4.5pt}]
\tikzstyle{fancyArrow} = [<->,very thick,decorate,decoration={snake,amplitude=.4mm,segment length=2mm,post length=1mm}]

	\coordinate (center) at (0,0);
	\draw[thick,dashed,rounded corners=.7cm] (-1.20,-1.) rectangle (1.20, 2.2);
	\node[colorbox=yellow] (cst) at (-3,1.7) {\strut cost};
	\node[colorbox=yellow, baseline=(cst.baseline)] (dst) at (3,1.7) {\strut distillate};
	\node[colorbox=magenta, baseline=(cst.baseline)] (rsc) at (0.0, 2.4) {\strut resource};
	\node[anchor=center, left] (ecost) at (-3,0.5) {$E_{\text{cost}}$};
	\node[anchor=center] (rho) at (0.0,0.5) {$\rho_{AB}$};
	\node[anchor=center, right] (distilled) at (3,0.5) {$\ketbra{\psi^{-}}{\psi^{-}}^{\otimes n}\otimes \rho_E$};
	\node[colorbox=white, right=of distilled, anchor=east] (edist) {$E_{dist}$};
	
	\node[anchor=center, left] (kcost) at (-3,-0.5) {$M_{\text{KeyCost}}$};
	\node[anchor=center] (prob) at (0.0,-0.5) {$P_{ABE}$};
	\node[anchor=center, right] (ckey) at (3,-0.5) {$P''_{AB}P_E$};
	\node[colorbox=white, right=of ckey.center] (skr) {$\keyrate{X}{Y}{Z}$};
	
	\node[thick, baseline=(cst.baseline)] (bul) at (0.0, 1.7) {\strut\textbullet};
%	\node[above=of bul] (res) {Resource (fixed)}; 
	

	\draw[vecArrow] (cst.east) to (bul.west);
	\draw[vecArrow] (bul.east) to (dst.west);
	\draw[->] (ecost) to node[above left]{$LOCC$} (rho); \draw[->] (rho) to node[above right]{$LOCC$} (distilled);
	\draw[->] (kcost) to node[above left]{$LOPC$} (prob); \draw[->] (prob) to node[above right]{$LOPC$} (ckey);

\end{tikzpicture}

    	\caption{Entanglement distillation and CKA utilize a resource (mixed state or probability distribution) to produce a distillate that factors out Eve.}
    	\label{Fig:intuition}
    \end{figure}
    
    To measure entanglement one might consider the least number of maximally entangled bipartite quantum states required to prepare a density matrix $\rho_{AB}$ by local operations and classical communication. 
    Similarly one might measure entanglement by the maximal number of singlets that can be obtained form $\rho$ by local operations and classical communication. 
    These measures are not the same. 
    The first is called \emph{entanglement cost} or \emph{entanglement of formation}; the second describes the \emph{distillate entanglement}. 
    We will discuss the classical counterparts, namely the \emph{information of formation} and the \emph{secret-key rate}.
    Figure \ref{Fig:intuition} illustrates the analogy between the resource, distillate and cost.
    This is part of the intuition that leads to bound information.
    
    