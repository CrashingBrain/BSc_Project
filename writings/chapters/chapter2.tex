\section{What is a shared key?}
    	\subsection{Common Secret} \label{commonsecret}%here?
	Let $X,Y,Z,S$ be random variables on the same range $\mathcal{X}$. Let $X$ be owned by Alice, $Y$ by Bob and $Z$ by Eve. Then
  \begin{align}
	 & P[X=Y=S] > 1 - \epsilon\label{eqn:common}\tag{common} \\ 
	 & I(X;Z) = 0 \: \wedge \: I(Y;Z) = 0 \label{eqn:secret}\tag{secret}
  \end{align}
for all $\epsilon > 0 $. \\
The first part defines the \textit{common} property: $X$ and $Y$ must be asymptotically the same. 
The second part states that the amount of information Eve can gather about $X$ and $Y$, through it's realization of $Z$, is $0$.
\section{The analogy with entanglement}
	The most fascinating feature that arises from quantum mechanics is quantum entanglement. As Einstein, Podolsky and Rosen pointed out almost a century ago \cite{einstein1935}, 
	the measurement of entangled states defies the classical understanding of the outcome. \\
	Given only one of two entangled quantum states \footnotemark ,  no information can be extracted from it. 
	To educe the information that lies in it, one has to have access to the whole system --- i.e. both the states. 
	This makes the two states \emph{inseparable}. 
	A complete (anti-)correlation exists then between maximally entangled states.\\
	 This is sufficient for the quantum state to be used as a variable between Alice and Bob to share a secret. 
	 One party can encode the message into quantum entangled states and later they will be able to read the same message.
	If Alice measures (ref. \ref{measurements}) $0$ on her part of the system, the part of the system owned by Bob will immediately become $1$.\\
	Quantum entanglement posses one more feature that classical correlation does not have: the monogamy of entanglement \cite{KW04}. 
	As Koashi and Winter state in their paper a fundamental difference is that classical correlation can be shared, while quantum entanglement can not. 
	This translates to the case where an eavesdropper Eve listens to the message exchange between Alice and Bob: in the classical communication there is no direct way for Alice nor Bob to know that Eve is listening (i.e. \textit{shares the correlation}), while in the second case Eve breaks the existing correlation between Alice and Bob.\\
This two aspects of quantum entanglement --- correaltion and monogamy --- give a valid framework for the establishment of a private channel between parties.
	
	\footnotetext{A quantum state in quantum mechanics describes a single and isolated quantum system. This can be for example an electron or a photon. For our purposes, a quantum state is always abstracted as a \emph{qubit} or multiple qubits, as described in appendix \ref{App:appendixB}}
   	 \begin{figure}[h]
			\centering
			%% Little schematics showing the origin of entanglement
%% from the linear theory of QM and tensor product

\begin{tikzpicture}[scale=0.6]
  \node[colorbox=red]                      (lin)  {Linearity};
  \node[colorbox=magenta, below=.5cm of lin] (tp)  {Tensor product};
  \node[colorbox=blue, below=.5cm of tp]   (en)  {Entanglement};
  \draw[conn=2pt, densely dotted] (tp) to (lin);
  \draw[conn=2pt, densely dotted] (en) to (tp);
\end{tikzpicture}
			\caption{origin of entanglement via linearity}
		\end{figure}
		
\section{Examples of key exchange}
	The majority of cryptographic systems used are built on computational complexity security. The so-called cryptographic functions are functions that are easy to compute in one way, but have a much higher complexity the other way round.
		\subsection{The Diffie-Hellman key exchange}
		% Explain only ho w the protocol works, what is based on, what are its bounds, how it can be attacked (ideally).
		% Don't dive into Maurer violations, that will be covered in chapter [3] (XX)
	
		A famous and widely used protocol for the exchange of cryptographic keys is the Diffie-Hellman key-exchange method.
	The whole process can be summarized in five basic steps:
	\begin{enumerate}
		\item Alice and Bob \emph{publicly} communicate and agree on two numbers, that will serve as basis for the computations.
		\item Each party generates \emph{locally} a personal and distinct secret ($s_A$ and $s_B$) without ever communicating it .
		\item They mix their own secret with the common agreed basis, producing a result $R_A$ and $R_B$. The mathematical properties of this operation make it so it is computational infeasible to go back and retrieve the secrets $s$ from $R$.
		\item Both parties exchange \emph{publicly} their result, so that they now posses the inseparable secret-base mixture of the other party.
		\item Each party applies again their secret but to the received mixture this time. The outputs are equal for Alice and Bob so they can use this result as a common secret to create a key.
	\end{enumerate}	 
	The parts exchanged over the public channel --- the ones that Eve knows --- are only the mutually agreed base and the two partial mixtures. 
	It can be proven that those two elements alone give no information about the complete final shared secret and that it is virtually impossible to obtain the correct final product with only those two.\\  
	
	The security in this method relies mainly in step 3. Here an action as $ R_A = g^{s_A} \bmod p $ is performed, where $g$ and $p$ are the public common basis agreed beforehand. 
	To get back to $s_A$  one will need to find the prime factors of $R_A$, which is a known hard problem. It is not impossible however. The difficulty of breaking this step is bounded only by the length of the number chosen one one side and the computational power available to the adversary on the other.
	\subsection{The BB84 protocol}
		
	\subsection{The One-time Pad}
		%this is not really a method to share a common secret, since you have 
		% to already start with a secret.. <- point that out!
		
		% example of utilization of the key after generation
		% info theoretical key exchanged with OTP over already existing channel
\section{A comparison between securities}
    %[7]
\section{The equivalent of CKA in QM}
    %[1]