
\section{Mathematical Framework (for QM)}
	
%	\begin{enumerate}
%	\item Dirac's braket notation
%	\item Inner/(Outer) product
%	\item Linear operator
%	\item Adjoints and Hermitian operators
%	\item Pauli matrices
%	\item Tensor Product and tensor space
%	\end{enumerate}
	
	
	\section{Inner product spaces}
	In standard vector notation we define the inner (scalar) product of complex vectors as
	$$ ( \vec{v}, \vec{w} ) =  \begin{pmatrix} \bar{v_1} & \bar{v_2}\end{pmatrix} \begin{pmatrix} w_1 \\ w_2 \end{pmatrix} = \begin{pmatrix} \bar{w_1} & \bar{w_2}\end{pmatrix} \begin{pmatrix} v_1 \\ v_2 \end{pmatrix} = ( \vec{w}, \vec{v} )^{\dagger}$$
	Where $\dagger$ represents the conjugate transpose.\\
%	This property is fundamental in the sense that it will allows us to go from a state space --- that can be many dimensional --- to a \textit{measurement} space, which assumes real values.\\ %0 and 1 in our case??
	
	It is important also to note that through the inner product of two vectors we also define the norm $\|\ket{v}\|  =  \sqrt{( \vec{v}, \vec{v} )} $.\\
	
	
	The outer product of two vectors, on the other hand, produces a matrix, with very important properties.  
	
	\section{Tensor product spaces}
	The tensor product $V\otimes W$ is an operation between vector spaces that combines every element of the first vector space and every element of the second vector space in a bigger vector space. Tensor product is linear and from its properties emerges the famous phenomenon of quantum entanglement, which simply is that not all vectors in $\H = V\otimes W$ can be divided into $\ket{v}\otimes\ket{w}$ with $\ket{v}\in V,\; \ket{w}\in W$. This will later be explained in the next section.\\
	Notation and abbreviation for the tensor product is 
	$$ \ket{v}\otimes\ket{w} = \ket{v}\ket{w} = \ket{v,w} = \ket{vw}$$
	It has the following properties:
	\begin{description}
		\item $\forall\ket{v}\in V ,\; \forall\ket{w}\in W, \; \forall z\in \mathbb{C}$	\\
					$ z(\ket{v}\otimes\ket{w}) = (z\ket{v})\otimes\ket{w} = \ket{v}\otimes(z\ket{w} $
		\item $\forall\ket{v_1},\ket{v_2}\in V ,\; \forall\ket{w}\in W$	\\
					$ (\ket{v_1} + \ket{v_2})\otimes\ket{w} = \ket{v_1w} + \ket{v_2w} $
		\item $\forall\ket{v}\in V ,\; \forall\ket{w}\in W, \; A:V\rightarrow V' \; B:W\rightarrow W'$	\\
					$ (A\otimes B) \left(\sum_i a_i \ket{v_i w_i} \right) = \sum_i a_i A\ket{v_i}\otimes B\ket{w_i} $
	\end{description}
	The inner product on $V$ and $W$ can be used to define (linearly) an inner product on $V\otimes W$.	
	
	

