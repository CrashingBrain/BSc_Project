
\section{Mathematical Framework (for QM)}
	
%	\begin{enumerate}
%	\item Dirac's braket notation
%	\item Inner/(Outer) product
%	\item Linear operator
%	\item Adjoints and Hermitian operators
%	\item Pauli matrices
%	\item Tensor Product and tensor space
%	\end{enumerate}
	
	
	\section{Inner product spaces}
	In standard vector notation we define the inner (scalar) product of complex vectors as
	$$ ( \vec{v}, \vec{w} ) =  \begin{pmatrix} \bar{v_1} & \bar{v_2}\end{pmatrix} \begin{pmatrix} w_1 \\ w_2 \end{pmatrix} = \begin{pmatrix} \bar{w_1} & \bar{w_2}\end{pmatrix} \begin{pmatrix} v_1 \\ v_2 \end{pmatrix} = ( \vec{w}, \vec{v} )^{\dagger}$$
	Where $\dagger$ represents the conjugate transpose.\\
%	This property is fundamental in the sense that it will allows us to go from a state space --- that can be many dimensional --- to a \textit{measurement} space, which assumes real values.\\ %0 and 1 in our case??
	
	It is important also to note that through the inner product of two vectors we also define the norm $\|\ket{v}\|  =  \sqrt{( \vec{v}, \vec{v} )} $.\\
	
	
	The outer product of two vectors, on the other hand, produces a matrix, with very important properties.  
	
	\section{Tensor product spaces}
	The tensor product $V\otimes W$ is an operation between vector spaces that combines every element of the first vector space and every element of the second vector space in a bigger vector space. Tensor product is linear and from its properties emerges the famous phenomenon of quantum entanglement, which simply is that not all vectors in $\H = V\otimes W$ can be divided into $\ket{v}\otimes\ket{w}$ with $\ket{v}\in V,\; \ket{w}\in W$. This will later be explained in the next section.\\
	Notation and abbreviation for the tensor product is 
	$$ \ket{v}\otimes\ket{w} = \ket{v}\ket{w} = \ket{v,w} = \ket{vw}$$
	It has the following properties:
	\begin{description}
		\item $\forall\ket{v}\in V ,\; \forall\ket{w}\in W, \; \forall z\in \mathbb{C}$	\\
					$ z(\ket{v}\otimes\ket{w}) = (z\ket{v})\otimes\ket{w} = \ket{v}\otimes(z\ket{w} $
		\item $\forall\ket{v_1},\ket{v_2}\in V ,\; \forall\ket{w}\in W$	\\
					$ (\ket{v_1} + \ket{v_2})\otimes\ket{w} = \ket{v_1w} + \ket{v_2w} $
		\item $\forall\ket{v}\in V ,\; \forall\ket{w}\in W, \; A:V\rightarrow V' \; B:W\rightarrow W'$	\\
					$ (A\otimes B) \left(\sum_i a_i \ket{v_i w_i} \right) = \sum_i a_i A\ket{v_i}\otimes B\ket{w_i} $
	\end{description}
	The inner product on $V$ and $W$ can be used to define (linearly) an inner product on $V\otimes W$.	
	
\section{Quantum Mechanics}
	%TODO Correct and reformulate this
	
	
	
	%TODO review this
	All pure states in QM are normalized vectors in $\H$.
	$$ \ket{\psi} \text{ is a state vector } \Rightarrow \ket{\psi}\in\H \text{ and }  \vert\bk{\psi}{\psi}\vert = 1$$
	This is instrumental in seeing them as probability vectors. Every linear operator has then to be unitary to maintain this property.\\
	A statistical mixture of states corresponds to a \emph{density matrix}, which is itself a new state. It is important to note that a mixture of probability of states is not the same thing as superposition of states. In the latter we don't have a measure of uncertainty of the state, meaning also that in theory we are always able to find a measurement basis that will always output the same result for that state. In the former, however, this is not possible given by the direct intrinsic uncertainty of the state.\\
	Density matrices have then the properties:
	$$ M = \rho = \sum_i p_i \ketbra{\psi_i}{\psi_i} = \sum_i p_i P_{\ket{\psi_i}} \text{  , where state }\ket{\psi_i}\text{ has probability } p_i $$ 
	$\rho$ is a positive, trace-1 operator meaning that $\Tr{\rho} = 1$ and all eigenvalues of $\rho$ are positive. Moreover $\rho$ is a linear combination of projectors $\proj{\psi_i}$ which makes $\rho\in\mathbf{P}(\H)$ a projector itself on the the Hilbert space.
	
		\subsection{Quantum Measurements}
		To get an actual value out of a qubit one has to \textit{measure} it. Measurement is, mathematically, a projection onto some chosen computational basis. The result for each base vector projection is then interpreted as a \emph{probability}. The state then changes after measurement, meaning for example that it will not retain it value as superposition any more.\\
		...\\
		
		If Alice has the state $\ket{psi_i}$ out of $i=1..n$ and all states are orthonormal, then Bob can find out what the choice of $i$ was.
		If the states are not orthonormal there is no quantum measurement capable of distinguishing the states. \\
		From this follows that if the states $\ket{\psi_1}$ and $\ket{\psi_2}$ are not orthogonal, then $\ket{\psi_2}$ has a component orthogonal to $\ket{\psi_1}$ but also a component parallel to it which will give probability not $0$ of measuring differently.
		
			\begin{xmpl}
      \begin{equation*} % environment for equation 
				Z = \begin{bmatrix} 1 & 0 \\ 0 & -1 \end{bmatrix} \quad P_{+1} = \proj{0} , \; P_{-1} = \proj{1}
      \end{equation*}
				Measurement on qubit $ \ket{\psi} = \frac{\ket0 + \ket1}{\sqrt{2}} $ has probability $p_{+1} = \bra{\psi}P_{+1}\ket{\psi} = \bk{\psi}{0}\bk{0}{\psi} = \frac{1}{2}$ and similarly $p_{-1} = \frac{1}{2}$
				\cite{NC10}
			\end{xmpl}
	

