% Motivation chapter
% contains also abstract explanation on theoretic secure key distribution

%\lipsum[3]
%The problem of secure and private communication between parties is as old as it can be complicated. 
%\\

There are nonetheless distributions of probabilities that can hold a value of privacy and can therefore used in cryptographic systems. These joint probabilities have the properties
\\

It has been noticed that these concepts of privacy also appear in nature and the strongest analogy comes from quantum mechanics.\footnotemark 
From this theory arises the famous \emph{quantum entanglement} that appears to be the equivalent of privacy in many ways.
Both phenomena are composed of correlations between known parties that no other person can access or copy. As summed in \cite{4H07} "If systems are in pure entangled state then at the same time (i) systems are correlated and (ii) no other system is correlated with them.". 
This can be seen as Alice and Bob holding a secret that Eve can not get to know.
\begin{table}[h]
	 \centering
	 	\begin{tabular}{ l | l}
	 		\textbf{quantum theory} & \textbf{classical information} \\ 
	 		\hline 
	 		quantum entanglement & secret classical correlations \\ 
	 		quantum communication & secret classical communication \\ 
	 		classical communication & public classical communication \\ 
	 		entanglement distillation & classical key agreement (CKA) \\ %TODO doubt here
	 		local actions & local actions \\ 
	 		bound entanglement & bound information ? \\
	 	\end{tabular} 
	 	\caption{Table showing key QM concepts and their analog in classical key agreement, following \cite{CP02}.
	 	\label{Tab:analogy}}
	 \end{table}
\\
\footnotetext{While those analogies are present in many sources, they can be found summed up in the paper by Collins and Popescu \cite{CP02}, which also shortly addresses the question of bound information. }

From table \ref{Tab:analogy} we see that some of the resources and operations of QM have a one-to-one analog in classical information theory. 
Such a close relation suggests that the two theories can be viewed together and to use one to better understand the other. 
It is important however to point out that quantum entanglement and its effects \emph{are not} a quantum manifestation of classical effects and one theory does not explain the other. \\
For example there is no known instance --- and it is believed to not exist -- of a classical correspondence to super-dense coding (a quantum effect). Other entities like a classical correspondence to bound entanglement, \emph{bound information}, are not excluded a priori and remain yet to be observed or disproved.

\paragraph*{Common Secret}
Intuitively a common secret is a piece of information (i.e. \textit{bits} of information) known to trusted parties --- for example Alice and Bob --- and to none else. 
In an environment where we allow the presence of an eavesdropper Eve, reaching such state is not always trivial. \\
There exist methods and protocols to generate such secrets, even from nothing, although they differ at different levels of secrecy. A notable one is the famous Diffie-Hellman method to generate a common cryptographic key \cite{DH76} .
%Add line about Shannon maybe?

A more formal and precise definition --- one that we may also use in calculations -- of a common secret is given later in section \ref{commonsecret}.
\\

\paragraph*{The Question}
\begin{itemize}
		\item Is there a tripartite probability $P_{ABE}$, that has some \textbf{cost} associated to it to create it, but has $0$ possible key bits distillable from it? 
\end{itemize}


	