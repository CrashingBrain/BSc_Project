% Motivation chapter
% contains also abstract explanation on theoretic secure key distribution

%\lipsum[3]

The goal of key exchange is to allow Alice and Bob to establish a secure, private channel for communication.
For two parties to communicate confidentially, it is first needed to share some secret key between them, so that each party can encrypt and decrypt the communication.
If two parties could not establish a secure --- as in information theoretic secure --- initial key exchange, they cannot communicate with absolute security without the risk of an eavesdropper Eve listening to them.
The ultimate goal of Alice and Bob is to achieve a level of privacy, such that no eavesdropper could have information about the communication, even partially.\\
%The tools or \emph{resources} to obtain such initial common secret can have many shapes and forms in modern telecommunication, often relying on computer science.\\

It has been noticed that these concepts of privacy also appear in nature and the strongest analogy comes from quantum mechanics.\footnotemark 
From this theory arises the famous \emph{quantum entanglement} that appears to be the equivalent of privacy in many ways.
Both phenomena are composed of correlations between known parties that no other person can access or copy. As summed in \cite{4H07} "If systems are in pure entangled state then at the same time (i) systems are correlated and (ii) no other system is correlated with them."
This can be seen as Alice and Bob holding a secret that Eve cannot get to know.
\begin{table}[h]
	 \centering
	 	\begin{tabular}{ l | l}
	 		\textbf{quantum theory} & \textbf{classical information} \\ 
	 		\hline 
	 		(pure) quantum entanglement & secret classical correlations \\ 
	 		quantum communication & secret classical communication \\ 
	 		classical communication & public classical communication \\ 
	 		entanglement distillation & classical key agreement (CKA) \\ 
	 		local actions & local actions \\ 
	 		bound entanglement & bound information ? \\
	 	\end{tabular} 
	 	\caption{Table showing key QM concepts and their analog in classical key agreement, following \cite{CP02}.
	 	\label{Tab:analogy}}
	 \end{table}
\\
\footnotetext{While those analogies are present in many sources, they can be found summed up in the paper by Collins and Popescu \cite{CP02}, which also shortly addresses the question of bound information. }

From table \ref{Tab:analogy} we see that some of the resources and operations of QM have a bijective analog in classical information theory. 
Such a (close) relation suggests that the two theories can be viewed together and to use one to better understand the other. 
It is important however to point out that quantum entanglement and its effects \emph{are not} a quantum manifestation of classical effects and one theory does not prove the other.
There are limitations to the correspondences. \\
For example there is no known instance --- and it is believed to not exist -- of a classical correspondence to super-dense coding (a quantum effect). Other entities like a classical correspondence to bound entanglement, \emph{bound information}, are not excluded a priori and remain yet to be observed or disproved.\\

	\begin{figure}[h!]
		\centering
		% Image of the big idea of the thesis
% discussed also for initial presentation
% should show the "track" that goes from
% QM(bound entanglement) to Classical bound information
\begin{tikzpicture}
\tikzstyle{vecArrow} = [thick, decoration={markings,mark=at position
   1 with {\arrow[semithick]{open triangle 60}}},
   double distance=1.4pt, shorten >= 5.5pt,
   preaction = {decorate},
   postaction = {draw,line width=1.4pt, white,shorten >= 4.5pt}]
\tikzstyle{fancyArrow} = [<->,very thick,decorate,decoration={snake,amplitude=.4mm,segment length=2mm,post length=1.5mm, pre length=1.5mm}]

  \draw[thick,dashed] (-5,-3.24) rectangle (-1, 3.24);
  \draw[thick, dashed] (1,-3.24) rectangle (5, 3.24);

  \node[below right, anchor=center] (en) at (-3, 3)  {\textbf{ENTANGLEMENT}};
  \node[below right, anchor=center] (pr) at (3, 3)  {\textbf{PRIVACY}};
  \node[below right, anchor=center] (dist) at (-3, 2)  {Entanglement distillation};
  \node[below right, anchor=center] (cka) at (3, 2)  {Classical key agreement};
  \node[below right, anchor=center] (be) at (-3, -2)  {Bound Entanglement};
  \node[below right, anchor=center] (bi) at (3, -2)  {Bound Information};
  \draw[vecArrow] (dist) to (be);
  \draw[vecArrow] (cka) to (bi);
  \draw[fancyArrow] (dist) to (cka);
  \draw[fancyArrow] (be) to (bi);

\end{tikzpicture}

		\caption{Certain aspects of quantum mechanics can be mapped to classical information theory.}
		\label{Fig:bigpicture}
	\end{figure}
An open question has remained over the years asking whether bound information exists in the classical regime.
Should this resource exists, it would be --- by definition --- unusable to generate any key from it.
Nevertheless the existence of a resource with bound information might lead to interesting bounds on information theoretical key exchange.
To summarize, we are interested in the following question:

\paragraph*{The Question}
\begin{itemize}
		\item[] Is there a tripartite probability $P_{ABE}$, corresponding to Alice and Bob wanting to establish a key unknown to Eve, that has some \emph{cost} associated to it to create it, but has $0$ possible key bits extractable from it? 
\end{itemize}


	