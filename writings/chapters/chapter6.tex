\label{ch:six}
Towards the end of the project we decided to implement a python module to do testing and analysis on some candidate probability distributions. The motivation for this was given mainly by the results showed again in \cite{RW03}. 
\section{Analysis design and goals}
    To visualize and motivate the scope of this analysis we expand Eq. \ref{eq:bounds} as follow
    \begin{equation}
    	\keyrate{X}{Y}{Z} \leq \redintrinfo{X}{Y}{Z} \leq \intrinfo{X}{Y}{Z} \leq I_{form} (X;Y|Z)
    \end{equation}
    including also the information of formation.\\
    As mentioned before, it is not well defined how to get direct values for the secret key rate and information of formation --- the fundamental quantities for key agreement --- since they are defined asymptotically over a \textit{possible} protocol. 
    For their bounds --- i.e. the central parts of the inequality --- we can obtain direct numerical values. \\
    The aim of this part of the project is then to take a candidate as given in Fig. \ref{Tab:candidate2} and map the values of the reduced and normal intrinsic information for various variation of said probability distribution.\\
    
    A good result we hope to obtain is a probability distribution for which the reduced measure tend to $0$, while the intrinsic information remains larger than $0$, maintaining a level of correlation between of Alice and Bob. 
    Due to the strictness of the bounds, this will constrain the value for the secret key rate down to (possibly) $0$, while keeping a non-zero key cost ($I_{\text{form}}$).
    This will lead to a new candidate for bound information.\\
    
    In order to perform analysis on those measures we firstly had to implement a library of modules that dealt with the probability and information theory aspects.
    Following criteria for separability of quantum states, a quantum mechanics module was also implemented to translate and later tests the distributions from the quantum to classical regime.\\
    The intrinsic information (and the reduced counterpart) is defined as an \emph{infimum} over the set of tripartite probability distributions. 
    To find a correct value it would require to solve an optimization problem. 
    The definition given in section \ref{intrininfo} however does not allow us to formulate the problem as convex, or, at least, not a trivial one.
    We decided then to adopt a Monte-Carlo method estimate them.\\
     
    For each step --- i.e. for each variation --- of the candidate base probability, the values of 
    \begin{itemize}
    	\item	mutual information
    	\item	intrinsic information
    	\item	reduced intrinsic information
    	\item	trace over quantum state witness given in \cite{DPS04}
    \end{itemize}
    are measured.
\section{Different noises analysis}
	Following the candidate distribution showed in Fig. \ref{Tab:candidate2} we also utilised a noise function that operates on the non-correlated part for Alice and Bob.
	The main feature that this noise offers is 
	\begin{figure}
		\begin{subfigure}{0.5\textwidth}
			%Table of probability distribution given in paper

	\begin{center}
		\begin{tabular}{|l r||c|c|c|c|}
		    \hline 
		    		 &	$X$ & $0$ & $1$ & $2$ & $3$ \\ 
		    $Y$ &		  &		&			&			&		\\
		    \hline 
		    \hline
		    $0$ &		   & $1/8$ & $1/8$ & $0$ & $0$ \\ 
		    \hline 
		    $1$ &		   & $1/8$ & $1/8$ & $0$ & $0$ \\ 
		    \hline 
		    $2$ &		   & $0$ & $0$ & $1/4$ & $0$ \\ 
		    \hline 
		    $3$ &		   & $0$ & $0$ & $0$ & $1/4$ \\ 
		    \hline 
		  \end{tabular} 
	\end{center}		

%\begin{table}[pos]
%\centering
%\caption{My caption}
%\label{my-label}
%\begin{tabular}{|l|l|l|l|l|l|}
%\hline
%\multicolumn{2}{|r|}{$X$} & \multirow{2}{*}{0} & \multirow{2}{*}{1} & \multirow{2}{*}{2} & \multirow{2}{*}{3} \\
%\multicolumn{2}{|l|}{Y}   &                    &                    &                    &                    \\ \hline
%\multicolumn{2}{|l|}{0}   & 1/8                & 1/8                & 0                  & 0                  \\ \hline
%\multicolumn{2}{|l|}{1}   & 1/8                & 1/8                & 0                  & 0                  \\ \hline
%\multicolumn{2}{|l|}{2}   & 0                  & 0                  & 1/4                & 0                  \\ \hline
%\multicolumn{2}{|l|}{3}   & 0                  & 0                  & 0                  & 1/4                \\ \hline
%\end{tabular}
%\end{table}
		
	    $$Z \equiv X + Y\; (mod\: 2)\; \text{ if } X,Y \in \{ 0,1\}$$ 
	    $$Z \equiv X\; (mod\: 2)\; \text{ if } X \in \{ 2,3\}$$ 
	    $$U \equiv \lfloor X/2 \rfloor $$
		\end{subfigure}
		\begin{subfigure}{0.5\textwidth}
			%Table of candidate distribution 1

		\begin{center}
		\begin{tabular}{|l r||c|c|c|c|}
		    \hline 
		    		 &	$X$ & $0$ & $1$ & $2$ & $3$ \\ 
		    $Y$ &		  &		&			&			&		\\
		    \hline 
		    \hline
		    $0$ &		   & $1/8$ & $1/8$ & $a$ & $a$ \\ 
		    \hline 
		    $1$ &		   & $1/8$ & $1/8$ & $a$ & $a$ \\ 
		    \hline 
		    $2$ &		   & $a$ & $a$ & $1/4$ & $0$ \\ 
		    \hline 
		    $3$ &		   & $a$ & $a$ & $0$ & $1/4$ \\ 
		    \hline 
		  \end{tabular} 
	\end{center}
	
	\begin{align*}
			Z \equiv&\: X + Y\; (mod\: 2)\; \text{ if } X,Y \in \{ 0,1\}\\
	    	Z \equiv&\: X\; (mod\: 2)\; \text{ if } X,Y \in \{ 2,3\}\\
	    	Z =&\: (X,Y) \text{ otherwise} 
	\end{align*}
	
%			$$Z \equiv X + Y\; (mod\: 2)\; \text{ if } X,Y \in \{ 0,1\}$$ 
%	    	$$Z \equiv X\; (mod\: 2)\; \text{ if } X,Y \in \{ 2,3\}$$ 
%	    	$$Z = (X,Y) \text{ otherwise} $$
		\end{subfigure}
	\end{figure}
\section{The problem with the reduced intrinsic information}
    